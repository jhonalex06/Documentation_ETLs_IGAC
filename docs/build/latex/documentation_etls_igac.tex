%% Generated by Sphinx.
\def\sphinxdocclass{report}
\documentclass[letterpaper,10pt,spanish]{sphinxmanual}
\ifdefined\pdfpxdimen
   \let\sphinxpxdimen\pdfpxdimen\else\newdimen\sphinxpxdimen
\fi \sphinxpxdimen=.75bp\relax

\PassOptionsToPackage{warn}{textcomp}
\usepackage[utf8]{inputenc}
\ifdefined\DeclareUnicodeCharacter
% support both utf8 and utf8x syntaxes
  \ifdefined\DeclareUnicodeCharacterAsOptional
    \def\sphinxDUC#1{\DeclareUnicodeCharacter{"#1}}
  \else
    \let\sphinxDUC\DeclareUnicodeCharacter
  \fi
  \sphinxDUC{00A0}{\nobreakspace}
  \sphinxDUC{2500}{\sphinxunichar{2500}}
  \sphinxDUC{2502}{\sphinxunichar{2502}}
  \sphinxDUC{2514}{\sphinxunichar{2514}}
  \sphinxDUC{251C}{\sphinxunichar{251C}}
  \sphinxDUC{2572}{\textbackslash}
\fi
\usepackage{cmap}
\usepackage[T1]{fontenc}
\usepackage{amsmath,amssymb,amstext}
\usepackage{babel}



\usepackage{times}
\expandafter\ifx\csname T@LGR\endcsname\relax
\else
% LGR was declared as font encoding
  \substitutefont{LGR}{\rmdefault}{cmr}
  \substitutefont{LGR}{\sfdefault}{cmss}
  \substitutefont{LGR}{\ttdefault}{cmtt}
\fi
\expandafter\ifx\csname T@X2\endcsname\relax
  \expandafter\ifx\csname T@T2A\endcsname\relax
  \else
  % T2A was declared as font encoding
    \substitutefont{T2A}{\rmdefault}{cmr}
    \substitutefont{T2A}{\sfdefault}{cmss}
    \substitutefont{T2A}{\ttdefault}{cmtt}
  \fi
\else
% X2 was declared as font encoding
  \substitutefont{X2}{\rmdefault}{cmr}
  \substitutefont{X2}{\sfdefault}{cmss}
  \substitutefont{X2}{\ttdefault}{cmtt}
\fi


\usepackage[Sonny]{fncychap}
\ChNameVar{\Large\normalfont\sffamily}
\ChTitleVar{\Large\normalfont\sffamily}
\usepackage{sphinx}

\fvset{fontsize=\small}
\usepackage{geometry}


% Include hyperref last.
\usepackage{hyperref}
% Fix anchor placement for figures with captions.
\usepackage{hypcap}% it must be loaded after hyperref.
% Set up styles of URL: it should be placed after hyperref.
\urlstyle{same}
\addto\captionsspanish{\renewcommand{\contentsname}{Contenido:}}

\usepackage{sphinxmessages}
\setcounter{tocdepth}{1}



\title{Documentation\_ETLs\_IGAC}
\date{23 de abril de 2020}
\release{0.0.1}
\author{Agencia de implementación}
\newcommand{\sphinxlogo}{\vbox{}}
\renewcommand{\releasename}{Versión}
\makeindex
\begin{document}

\ifdefined\shorthandoff
  \ifnum\catcode`\=\string=\active\shorthandoff{=}\fi
  \ifnum\catcode`\"=\active\shorthandoff{"}\fi
\fi

\pagestyle{empty}
\sphinxmaketitle
\pagestyle{plain}
\sphinxtableofcontents
\pagestyle{normal}
\phantomsection\label{\detokenize{index::doc}}


Este documento tiene como objetivo proporcionar la facilidad de usar, estudiar, compartir y mejorar los ETL del sistema COBOL
y el Sistema Nacional Catastral (IGAC), para esto se divide el documento en dos partes fundamentales, la primera, hace referencia
a la ejecución de cada uno de los ETL desarrollados y en la segunda se encuentra una descripción de los aspectos técnicos de la
estructura del modelo.

Los ETL del sistema de COBOL y el Sistema Nacional Catastral fueron desarrollados en el software QGIS debido a su facilidad para el
manejo de diferentes fuentes de información y además de la posibilidad de integrar la información geográfica con la información alfanumérica.
Para la generación de estos ETL fue necesario el uso de los diferentes insumos proporcionados por cada uno de los sistemas al interior de IGAC,
\begin{quote}

los cuales se mencionan al comienzo de cada ETL.
\end{quote}


\chapter{ETL Cobol}
\label{\detokenize{ETL Cobol:etl-cobol}}\label{\detokenize{ETL Cobol::doc}}
Esta sección describe los insumos


\section{Insumos necesarios para ejecutar el ETL}
\label{\detokenize{ETL_Cobol/supplies:insumos-necesarios-para-ejecutar-el-etl}}\label{\detokenize{ETL_Cobol/supplies::doc}}
Para la ejecución del ETL de COBOL es necesario el uso de cinco (5) insumos con extensión .lis, los cuales se distribuyen de la siguiente manera, cuatro (4) archivos obligatorios y uno (1) opcional, los cuales se mencionan a continuación:
\begin{itemize}
\item {} 
Blo.lis \sphinxhyphen{} (Opcional)

\item {} 
Uni.lis\sphinxhyphen{} (Obligatorio)

\item {} 
Ter.lis\sphinxhyphen{} (Obligatorio)

\item {} 
Pro.lis\sphinxhyphen{} (Obligatorio)

\end{itemize}


\section{Estructura de los insumos}
\label{\detokenize{ETL_Cobol/Structure:estructura-de-los-insumos}}\label{\detokenize{ETL_Cobol/Structure::doc}}

\subsection{Blo.lis}
\label{\detokenize{ETL_Cobol/Structure:blo-lis}}\begin{itemize}
\item {} 
DEPART\sphinxhyphen{}BL

\item {} 
MUNICI\sphinxhyphen{}BL

\item {} 
NUMPRED\sphinxhyphen{}BL

\item {} 
ESTADO\sphinxhyphen{}BL

\item {} 
ENT\sphinxhyphen{}EMISORA\sphinxhyphen{}BL

\item {} 
ANO\sphinxhyphen{}SIST\sphinxhyphen{}ALERTA\sphinxhyphen{}BL

\item {} 
MES\sphinxhyphen{}SIST\sphinxhyphen{}ALERTA\sphinxhyphen{}BL

\item {} 
DIA\sphinxhyphen{}SIST\sphinxhyphen{}ALERTA\sphinxhyphen{}BL

\end{itemize}


\subsection{Uni.lis}
\label{\detokenize{ETL_Cobol/Structure:uni-lis}}\begin{itemize}
\item {} 
UNI\sphinxhyphen{}CDEP

\item {} 
UNI\sphinxhyphen{}CMUN

\item {} 
UNI\sphinxhyphen{}NUMPRED

\item {} 
UNI\sphinxhyphen{}TICA\sphinxhyphen{}E2

\item {} 
UNI\sphinxhyphen{}MATRICULA

\item {} 
UNI\sphinxhyphen{}DECO\sphinxhyphen{}E2

\item {} 
UNI\sphinxhyphen{}DIRECCION

\item {} 
UNI\sphinxhyphen{}HECTAREA

\item {} 
UNI\sphinxhyphen{}METROSTE

\end{itemize}


\subsection{Ter.lis}
\label{\detokenize{ETL_Cobol/Structure:ter-lis}}\begin{itemize}
\item {} 
DEP\sphinxhyphen{}TER

\item {} 
MUN\sphinxhyphen{}TER

\item {} 
NUM\sphinxhyphen{}PREDIAL\sphinxhyphen{}TER

\item {} 
PIS21\sphinxhyphen{}TER

\item {} 
DES21\sphinxhyphen{}TER

\item {} 
PUN21\sphinxhyphen{}TER

\item {} 
ARE21\sphinxhyphen{}TER

\item {} 
PIS22\sphinxhyphen{}TER

\item {} 
DES22\sphinxhyphen{}TER

\item {} 
PUN22\sphinxhyphen{}TER

\item {} 
ARE22\sphinxhyphen{}TER

\item {} 
PIS23\sphinxhyphen{}TER

\item {} 
DES23\sphinxhyphen{}TER

\item {} 
PUN23\sphinxhyphen{}TER

\item {} 
ARE23\sphinxhyphen{}TER

\end{itemize}


\subsection{Pro.lis}
\label{\detokenize{ETL_Cobol/Structure:pro-lis}}\begin{itemize}
\item {} 
TIPO\sphinxhyphen{}DOC\sphinxhyphen{}PRO

\item {} 
NUMERO\sphinxhyphen{}DOC\sphinxhyphen{}PRO

\item {} 
NOMBRE\sphinxhyphen{}PROP\sphinxhyphen{}PRO

\end{itemize}


\section{Procedimiento para ejecutar el ETL}
\label{\detokenize{ETL_Cobol/Process:procedimiento-para-ejecutar-el-etl}}\label{\detokenize{ETL_Cobol/Process::doc}}\begin{enumerate}
\sphinxsetlistlabels{\arabic}{enumi}{enumii}{}{.}%
\item {} 
Abrir QGIS.

\item {} 
Instalar el Asistente LADM\sphinxhyphen{}COL.

\item {} 
Seleccionar LADM\sphinxhyphen{}COL \sphinxhyphen{}\textgreater{} Administración de datos \sphinxhyphen{}\textgreater{} Crear estructura LADM\sphinxhyphen{}COL.

\item {} 
Realizamos clic en la opción, Configurar conexión.

\item {} 
Diligenciamos los datos de Host, Puerto, Usuario, Contraseña y realizamos clic en Refrescar bases de datos y esquemas.

\item {} 
Seleccionamos la base de datos y el esquema en donde queremos crear la estructura de LADM\sphinxhyphen{}COL, por último, clic en Aceptar.

\sphinxincludegraphics{{config_database1}.png}

\item {} 
En la interfaz que se despliega, seleccionamos el SRC de nuestros datos, el modelo Datos\_Gestor\_Catastral\_V2\_9\_6 y clic en Crear estructura LADM\sphinxhyphen{}COL.

\sphinxincludegraphics{{create_structure1}.png}

\item {} 
Una vez se ha creado la estructura de Datos\_Gestor\_Catastral, se da clic en Cerrar.

\item {} 
Seleccionar LADM\sphinxhyphen{}COL \sphinxhyphen{}\textgreater{} Gestión de insumos de datos \sphinxhyphen{}\textgreater{} Crear estructura LADM\sphinxhyphen{}COL.

\item {} 
Seleccionar ETL para datos Cobol \sphinxhyphen{}\textgreater{} Clic en Next

\sphinxincludegraphics{{initial_etl_snc1}.png}

\item {} 
Seguido es necesario cargar cada uno de los archivos con extensión .lis y definir la base de datos destino a partir del botón configurar conexión \sphinxhyphen{}\textgreater{} seleccionamos el esquema en el que deseamos crear los datos.

\item {} 
Por último, clic en Ejecutar ETL.

\sphinxincludegraphics{{supplies_etl_snc1}.png}

\item {} 
Por último, se obtiene un resumen de los datos transferidos al modelo LADM\sphinxhyphen{}COL y clic en finalizar.

\sphinxincludegraphics{{result_etl_snc1}.png}

\end{enumerate}


\section{Estructura del modelo}
\label{\detokenize{ETL_Cobol/Model:estructura-del-modelo}}\label{\detokenize{ETL_Cobol/Model::doc}}

\subsection{Funciones involucradas}
\label{\detokenize{ETL_Cobol/Model:funciones-involucradas}}\begin{itemize}
\item {} 
Corregir geometrías.

\item {} 
Calculadora de campos.

\item {} 
Eliminar vértices duplicados.

\item {} 
Extraer por atributo.

\item {} 
Insertar registros a la capa.

\item {} 
Rehacer campos.

\item {} 
Unir atributos por valor de campo.

\item {} 
Unir capas vectoriales.

\end{itemize}


\subsection{Flujo del diligenciamiento de tablas en el modelo LADM\sphinxhyphen{}COL}
\label{\detokenize{ETL_Cobol/Model:flujo-del-diligenciamiento-de-tablas-en-el-modelo-ladm-col}}

\subsection{GC\_Predio\_Catastro}
\label{\detokenize{ETL_Cobol/Model:gc-predio-catastro}}\begin{enumerate}
\sphinxsetlistlabels{\arabic}{enumi}{enumii}{}{.}%
\item {} 
Extraer por atributo.

\item {} 
Rehacer campos.

\item {} 
Insertar registros a la capa.

\sphinxincludegraphics{{gc_predio1}.png}

\end{enumerate}


\subsection{GC\_Terreno}
\label{\detokenize{ETL_Cobol/Model:gc-terreno}}\begin{enumerate}
\sphinxsetlistlabels{\arabic}{enumi}{enumii}{}{.}%
\item {} 
Unir capas vectoriales.

\item {} 
Corregir geometrías.

\item {} 
Unir atributos por valor de campo.

\item {} 
Rehacer campos.

\item {} 
Eliminar vértices duplicados.

\item {} 
Insertar registros a la capa.

\sphinxincludegraphics{{gc_terreno1}.png}

\end{enumerate}


\subsection{GC\_Comisiones\_Terreno}
\label{\detokenize{ETL_Cobol/Model:gc-comisiones-terreno}}\begin{enumerate}
\sphinxsetlistlabels{\arabic}{enumi}{enumii}{}{.}%
\item {} 
Unir capas vectoriales.

\item {} 
Corregir geometrías.

\item {} 
Unir atributos por valor de campo.

\item {} 
Unir capas vectoriales.

\item {} 
Rehacer campos.

\item {} 
Eliminar vértices duplicados.

\item {} 
Insertar registros a la capa.

\sphinxincludegraphics{{gc_comisiones_terreno1}.png}

\end{enumerate}


\subsection{GC\_Direccion}
\label{\detokenize{ETL_Cobol/Model:gc-direccion}}
Nota: Comenzando de izquierda a derecha
\begin{enumerate}
\sphinxsetlistlabels{\arabic}{enumi}{enumii}{}{.}%
\item {} 
Unir capas vectoriales.

\item {} 
Unir atributos por valor de campo.

\item {} 
Unir atributos por valor de campo.

\item {} 
Unir atributos por valor de campo.

\item {} 
Unir atributos por valor de campo.

\item {} 
Unir atributos por valor de campo.

\item {} 
Rehacer campos.

\item {} 
Rehacer campos.

\item {} 
Unir capas vectoriales.

\item {} 
Insertar registros a la capa.

\sphinxincludegraphics{{gc_direccion1}.png}

\end{enumerate}


\subsection{GC\_Construccion}
\label{\detokenize{ETL_Cobol/Model:gc-construccion}}\begin{enumerate}
\sphinxsetlistlabels{\arabic}{enumi}{enumii}{}{.}%
\item {} 
Unir capas vectoriales.

\item {} 
Corregir geometrías.

\item {} 
Unir atributos por valor de campo.

\item {} 
Rehacer campos.

\item {} 
Extraer por atributo.

\item {} 
Insertar registros a la capa.

\sphinxincludegraphics{{gc_construccion1}.png}

\end{enumerate}


\subsection{GC\_Comisiones\_Construccion}
\label{\detokenize{ETL_Cobol/Model:gc-comisiones-construccion}}\begin{enumerate}
\sphinxsetlistlabels{\arabic}{enumi}{enumii}{}{.}%
\item {} 
Unir capas vectoriales.

\item {} 
Corregir geometrías.

\item {} 
Unir atributos por valor de campo.

\item {} 
Rehacer campos.

\item {} 
Extraer por atributo.

\item {} 
Insertar registros a la capa.

\sphinxincludegraphics{{gc_comisiones_construccion1}.png}

\end{enumerate}


\subsection{GC\_Unidad\_Construccion}
\label{\detokenize{ETL_Cobol/Model:gc-unidad-construccion}}\begin{enumerate}
\sphinxsetlistlabels{\arabic}{enumi}{enumii}{}{.}%
\item {} 
Unir capas vectoriales.

\item {} 
Corregir geometrías.

\item {} 
Extraer por atributo.

\item {} 
Unir atributos por valor de campo.

\item {} 
Unir atributos por valor de campo.

\item {} 
Rehacer campos.

\item {} 
Extraer por atributo.

\item {} 
Unir capas vectoriales.

\item {} 
Unir capas vectoriales.

\item {} 
Corregir geometrías.

\item {} 
Extraer por atributo.

\item {} 
Unir capas vectoriales.

\item {} 
Rehacer campos.

\item {} 
Unir capas vectoriales.

\item {} 
Insertar registros a la capa.

\sphinxincludegraphics{{gc_unidad_construccion1}.png}

\end{enumerate}


\subsection{GC\_Comisiones\_Unidad\_Construccion}
\label{\detokenize{ETL_Cobol/Model:gc-comisiones-unidad-construccion}}\begin{enumerate}
\sphinxsetlistlabels{\arabic}{enumi}{enumii}{}{.}%
\item {} 
Unir capas vectoriales.

\item {} 
Corregir geometrías.

\item {} 
Extraer por atributo.

\item {} 
Unir atributos por valor de campo.

\item {} 
Unir atributos por valor de campo.

\item {} 
Rehacer campos.

\item {} 
Extraer por atributo.

\item {} 
Unir capas vectoriales.

\item {} 
Unir capas vectoriales.

\item {} 
Corregir geometrías.

\item {} 
Extraer por atributo.

\item {} 
Unir capas vectoriales.

\item {} 
Rehacer campos.

\item {} 
Unir capas vectoriales.

\item {} 
Insertar registros a la capa.

\sphinxincludegraphics{{gc_comisiones_construccion1}.png}

\end{enumerate}


\subsection{GC\_Copropiedad}
\label{\detokenize{ETL_Cobol/Model:gc-copropiedad}}\begin{enumerate}
\sphinxsetlistlabels{\arabic}{enumi}{enumii}{}{.}%
\item {} 
Calculadora de campos.

\item {} 
Unir atributos por valor de campo.

\item {} 
Unir atributos por valor de campo.

\item {} 
Rehacer campos.

\item {} 
Insertar registros a la capa.

\sphinxincludegraphics{{gc_copropiedad1}.png}

\end{enumerate}


\subsection{GC\_Datos\_PH\_Condominio}
\label{\detokenize{ETL_Cobol/Model:gc-datos-ph-condominio}}\begin{enumerate}
\sphinxsetlistlabels{\arabic}{enumi}{enumii}{}{.}%
\item {} 
Unir atributos por valor de campo.

\item {} 
Unir atributos por valor de campo.

\item {} 
Unir atributos por valor de campo.

\item {} 
Rehacer campos.

\item {} 
Insertar registros a la capa.

\sphinxincludegraphics{{gc_datos_ph_condominio1}.png}

\end{enumerate}


\subsection{GC\_Propietario}
\label{\detokenize{ETL_Cobol/Model:gc-propietario}}\begin{enumerate}
\sphinxsetlistlabels{\arabic}{enumi}{enumii}{}{.}%
\item {} 
Unir atributos por valor de campo.

\item {} 
Unir atributos por valor de campo.

\item {} 
Unir atributos por valor de campo.

\item {} 
Rehacer campos.

\item {} 
Insertar registros a la capa.

\sphinxincludegraphics{{gc_propietario1}.png}

\end{enumerate}


\subsection{GC\_Perimetro}
\label{\detokenize{ETL_Cobol/Model:gc-perimetro}}\begin{enumerate}
\sphinxsetlistlabels{\arabic}{enumi}{enumii}{}{.}%
\item {} 
Corregir geometrías.

\item {} 
Rehacer campos.

\item {} 
Eliminar vértices duplicados.

\item {} 
Insertar registros a la capa.

\sphinxincludegraphics{{gc_perimetro1}.png}

\end{enumerate}


\subsection{GC\_Vereda}
\label{\detokenize{ETL_Cobol/Model:gc-vereda}}\begin{enumerate}
\sphinxsetlistlabels{\arabic}{enumi}{enumii}{}{.}%
\item {} 
Corregir geometrías.

\item {} 
Rehacer campos.

\item {} 
Eliminar vértices duplicados.

\item {} 
Insertar registros a la capa.

\sphinxincludegraphics{{gc_vereda1}.png}

\end{enumerate}


\subsection{GC\_Manzana}
\label{\detokenize{ETL_Cobol/Model:gc-manzana}}\begin{enumerate}
\sphinxsetlistlabels{\arabic}{enumi}{enumii}{}{.}%
\item {} 
Corregir geometrías.

\item {} 
Rehacer campos.

\item {} 
Eliminar vértices duplicados.

\item {} 
Insertar registros a la capa.

\sphinxincludegraphics{{gc_manzana1}.png}

\end{enumerate}


\subsection{GC\_Barrio}
\label{\detokenize{ETL_Cobol/Model:gc-barrio}}\begin{enumerate}
\sphinxsetlistlabels{\arabic}{enumi}{enumii}{}{.}%
\item {} 
Corregir geometrías.

\item {} 
Rehacer campos.

\item {} 
Eliminar vértices duplicados.

\item {} 
Insertar registros a la capa.

\sphinxincludegraphics{{gc_barrio1}.png}

\end{enumerate}


\subsection{GC\_Sector\_Urbano}
\label{\detokenize{ETL_Cobol/Model:gc-sector-urbano}}\begin{enumerate}
\sphinxsetlistlabels{\arabic}{enumi}{enumii}{}{.}%
\item {} 
Corregir geometrías.

\item {} 
Rehacer campos.

\item {} 
Eliminar vértices duplicados.

\item {} 
Insertar registros a la capa.

\sphinxincludegraphics{{gc_sector_urbano1}.png}

\end{enumerate}


\subsection{GC\_Sector\_Rural}
\label{\detokenize{ETL_Cobol/Model:gc-sector-rural}}\begin{enumerate}
\sphinxsetlistlabels{\arabic}{enumi}{enumii}{}{.}%
\item {} 
Corregir geometrías.

\item {} 
Rehacer campos.

\item {} 
Eliminar vértices duplicados.

\item {} 
Insertar registros a la capa.

\sphinxincludegraphics{{gc_sector_rural1}.png}

\end{enumerate}


\section{Estructura del ETL SNC en repositorio}
\label{\detokenize{ETL_Cobol/Repository:estructura-del-etl-snc-en-repositorio}}\label{\detokenize{ETL_Cobol/Repository::doc}}
\sphinxhref{https://github.com/AgenciaImplementacion/Asistente-LADM-COL}{Repositorio}

\sphinxhref{https://github.com/AgenciaImplementacion/Asistente-LADM-COL/tree/master/asistente\_ladm\_col/gui/supplies}{Lógica de la interfaz}
\begin{itemize}
\item {} 
cobol\_data\_sources\_widget.py

\item {} 
wiz\_supplies\_etl.py

\end{itemize}

\sphinxhref{https://github.com/AgenciaImplementacion/Asistente-LADM-COL/tree/master/asistente\_ladm\_col/resources/etl}{Recursos}
\begin{itemize}
\item {} 
Blo.lis

\end{itemize}

\sphinxhref{https://github.com/AgenciaImplementacion/Asistente-LADM-COL/tree/master/asistente\_ladm\_col/ui/supplies}{Interfaces}
\begin{itemize}
\item {} 
cobol\_data\_source\_widget.ui

\item {} 
wiz\_supplies\_etl.ui

\end{itemize}


\chapter{ETL SNC}
\label{\detokenize{ETL SNC:etl-snc}}\label{\detokenize{ETL SNC::doc}}
Esta sección describe los insumos


\section{Insumos necesarios para ejecutar el ETL}
\label{\detokenize{ETL_SNC/supplies:insumos-necesarios-para-ejecutar-el-etl}}\label{\detokenize{ETL_SNC/supplies::doc}}
Para la ejecución del ETL del SNC es necesario el uso de diez (10) insumos con extensión .csv, los cuales se distribuyen
de la siguiente manera, síes (6) archivos obligatorios y cuatro (4) opcionales, los cuales se mencionan a continuación:
\begin{itemize}
\item {} 
Predio\_bloqueo.csv \sphinxhyphen{} (Opcional)

\item {} 
Predio.csv \sphinxhyphen{} (Obligatorio)

\item {} 
Persona.csv \sphinxhyphen{} (Obligatorio)

\item {} 
Persona\_predio.csv \sphinxhyphen{} (Obligatorio)

\item {} 
Unidad\_construccion.csv \sphinxhyphen{} (Obligatorio)

\item {} 
Unidad\_construccion\_comp.csv \sphinxhyphen{} (Obligatorio)

\item {} 
Ficha\_matriz.csv \sphinxhyphen{} (Opcional)

\item {} 
Ficha\_matriz\_predio.csv \sphinxhyphen{} (Opcional)

\item {} 
Ficha\_matriz\_torre.csv \sphinxhyphen{} (Opcional)

\end{itemize}


\section{Estructura de los insumos}
\label{\detokenize{ETL_SNC/Structure:estructura-de-los-insumos}}\label{\detokenize{ETL_SNC/Structure::doc}}

\subsection{Predio\_bloqueo.csv}
\label{\detokenize{ETL_SNC/Structure:predio-bloqueo-csv}}\begin{itemize}
\item {} 
FECHA INICIO BLOQUEO

\item {} 
TIPO BLOQUEO

\item {} 
TIPO DESBLOQUEO

\item {} 
ENTIDAD BLOQUEO

\end{itemize}


\subsection{Predio.csv}
\label{\detokenize{ETL_SNC/Structure:predio-csv}}\begin{itemize}
\item {} 
ID

\item {} 
NUMERO\_PREDIAL

\item {} 
NUMERO\_PREDIAL\_ANTERIOR

\item {} 
CIRCULO\_REGISTRAL

\item {} 
NUMERO\_REGISTRO

\item {} 
TIPO

\item {} 
CONDICION\_PROPIEDAD

\item {} 
DESTINO

\end{itemize}


\subsection{Predio\_direccion.csv}
\label{\detokenize{ETL_SNC/Structure:predio-direccion-csv}}\begin{itemize}
\item {} 
ID

\item {} 
PREDIO\_ID

\item {} 
DIRECCION

\item {} 
PRINCIPAL

\end{itemize}


\subsection{Persona.csv}
\label{\detokenize{ETL_SNC/Structure:persona-csv}}\begin{itemize}
\item {} 
ID

\item {} 
TIPO\_IDENTIFICACION

\item {} 
NUMERO\_IDENTIFICACION

\item {} 
DIGITO\_VERIFICACION

\item {} 
PRIMER\_NOMBRE

\item {} 
SEGUNDO\_NOMBRE

\item {} 
PRIMER\_APELLIDO

\item {} 
SEGUNDO\_APELLIDO

\item {} 
RAZON\_SOCIAL

\end{itemize}


\subsection{Persona\_predio.csv}
\label{\detokenize{ETL_SNC/Structure:persona-predio-csv}}\begin{itemize}
\item {} 
ID

\item {} 
PREDIO\_ID

\item {} 
PERSONA\_ID

\end{itemize}


\subsection{Unidad\_construccion.csv}
\label{\detokenize{ETL_SNC/Structure:unidad-construccion-csv}}\begin{itemize}
\item {} 
ID

\item {} 
PREDIO\_ID

\item {} 
UNIDAD

\item {} 
TIPO\_DOMINIO

\item {} 
TIPO\_CONSTRUCCION

\item {} 
PISO\_UBICACIÓN

\item {} 
TOTAL\_HABITACIONES

\item {} 
TOTAL\_BANIOS

\item {} 
TOTAL\_LOCALES

\item {} 
TOTAL\_PISOS\_UNIDAD

\item {} 
USO\_ID

\item {} 
ANIO\_CONSTRUCCION

\item {} 
TOTAL\_PUNTAJE

\item {} 
AREA\_CONSTRUIDA

\end{itemize}


\subsection{Unidad\_construccion\_comp.csv}
\label{\detokenize{ETL_SNC/Structure:unidad-construccion-comp-csv}}\begin{itemize}
\item {} 
ID

\item {} 
UNIDAD\_CONSTRUCCION\_ID

\item {} 
COMPONENTE

\item {} 
ELEMENTO\_CALIFICACION

\item {} 
DETALLE\_CALIFICACION

\item {} 
PUNTOS

\end{itemize}


\subsection{Ficha\_matriz.csv}
\label{\detokenize{ETL_SNC/Structure:ficha-matriz-csv}}\begin{itemize}
\item {} 
ID

\item {} 
PREDIO\_ID

\item {} 
AREA\_TOTAL\_CONSTRUIDA\_COMUN

\item {} 
AREA\_TOTAL\_CONSTRUIDA\_PRIVADA

\item {} 
AREA\_TOTAL\_TERRENO\_COMUN

\item {} 
AREA\_TOTAL\_TERRENO\_PRIVADA

\item {} 
VALOR\_TOTAL\_AVALUO\_CATASTRAL

\item {} 
TOTAL\_UNIDADES\_PRIVADAS

\item {} 
TOTAL\_UNIDADES\_SOTANOS

\end{itemize}


\subsection{Ficha\_matriz\_predio.csv}
\label{\detokenize{ETL_SNC/Structure:ficha-matriz-predio-csv}}\begin{itemize}
\item {} 
ID

\item {} 
FICHA\_MATRIZ\_ID

\item {} 
NUMERO\_PREDIAL

\item {} 
COEFICIENTE

\item {} 
CONSECUTIVO\_UNIDAD

\end{itemize}


\subsection{Ficha\_matriz\_torre.csv}
\label{\detokenize{ETL_SNC/Structure:ficha-matriz-torre-csv}}\begin{itemize}
\item {} 
ID

\item {} 
FICHA\_MATRIZ\_ID

\item {} 
TORRE

\item {} 
PISOS

\item {} 
UNIDADES\_PRIVADAS

\item {} 
SOTANOS

\item {} 
UNIDADES\_SOTANOS

\end{itemize}


\section{Procedimiento para ejecutar el ETL}
\label{\detokenize{ETL_SNC/Process:procedimiento-para-ejecutar-el-etl}}\label{\detokenize{ETL_SNC/Process::doc}}\begin{enumerate}
\sphinxsetlistlabels{\arabic}{enumi}{enumii}{}{.}%
\item {} 
Abrir QGIS.

\item {} 
Instalar el Asistente LADM\sphinxhyphen{}COL.

\item {} 
Seleccionar LADM\sphinxhyphen{}COL \sphinxhyphen{}\textgreater{} Administración de datos \sphinxhyphen{}\textgreater{} Crear estructura LADM\sphinxhyphen{}COL.

\item {} 
Realizamos clic en la opción, Configurar conexión.

\item {} 
Diligenciamos los datos de Host, Puerto, Usuario, Contraseña y realizamos clic en Refrescar bases de datos y esquemas.

\item {} 
Seleccionamos la base de datos y el esquema en donde queremos crear la estructura de LADM\sphinxhyphen{}COL, por último, clic en Aceptar.

\sphinxincludegraphics{{config_database}.png}

\item {} 
En la interfaz que se despliega, seleccionamos el SRC de nuestros datos, el modelo Datos\_Gestor\_Catastral\_V2\_9\_6 y clic en Crear estructura LADM\sphinxhyphen{}COL.

\sphinxincludegraphics{{create_structure}.png}

\item {} 
Una vez se ha creado la estructura de Datos\_Gestor\_Catastral, se da clic en Cerrar.

\item {} 
Seleccionar LADM\sphinxhyphen{}COL \sphinxhyphen{}\textgreater{} Gestión de insumos de datos \sphinxhyphen{}\textgreater{} Crear estructura LADM\sphinxhyphen{}COL.

\item {} 
Seleccionar ETL para datos SNC \sphinxhyphen{}\textgreater{} Clic en Next.

\sphinxincludegraphics{{initial_etl_snc}.png}

\item {} 
Seguido es necesario cargar cada uno de los archivos con extensión .csv y definir la base de datos destino a partir del botón configurar conexión \sphinxhyphen{}\textgreater{} seleccionamos el esquema en el que deseamos crear los datos.

\item {} 
Por último, clic en Ejecutar ETL.

\sphinxincludegraphics{{supplies_etl_snc}.png}

\item {} 
Por último, se obtiene un resumen de los datos transferidos al modelo LADM\sphinxhyphen{}COL y clic en finalizar.

\sphinxincludegraphics{{result_etl_snc}.png}

\end{enumerate}


\section{Estructura del modelo}
\label{\detokenize{ETL_SNC/Model:estructura-del-modelo}}\label{\detokenize{ETL_SNC/Model::doc}}

\subsection{Funciones involucradas}
\label{\detokenize{ETL_SNC/Model:funciones-involucradas}}\begin{itemize}
\item {} 
Corregir geometrías.

\item {} 
Calculadora de campos.

\item {} 
Eliminar vértices duplicados.

\item {} 
Extraer por atributo.

\item {} 
Insertar registros a la capa.

\item {} 
Rehacer campos.

\item {} 
Unir atributos por valor de campo.

\item {} 
Unir capas vectoriales.

\end{itemize}


\subsection{Flujo del diligenciamiento de tablas en el modelo LADM\sphinxhyphen{}COL}
\label{\detokenize{ETL_SNC/Model:flujo-del-diligenciamiento-de-tablas-en-el-modelo-ladm-col}}

\subsection{GC\_Predio\_Catastro}
\label{\detokenize{ETL_SNC/Model:gc-predio-catastro}}\begin{enumerate}
\sphinxsetlistlabels{\arabic}{enumi}{enumii}{}{.}%
\item {} 
Calculadora de campos.

\item {} 
Calculadora de campos.

\item {} 
Rehacer campos.

\item {} 
Insertar registros a la capa.

\item {} 
Calculadora de campos.

\item {} 
Calculadora de campos.

\item {} 
Rehacer campos.

\item {} 
Insertar registros a la capa.

\sphinxincludegraphics{{gc_predio}.png}

\end{enumerate}


\subsection{GC\_Terreno}
\label{\detokenize{ETL_SNC/Model:gc-terreno}}\begin{enumerate}
\sphinxsetlistlabels{\arabic}{enumi}{enumii}{}{.}%
\item {} 
Unir capas vectoriales.

\item {} 
Corregir geometrías.

\item {} 
Unir atributos por valor de campo.

\item {} 
Unir atributos por valor de campo.

\item {} 
Rehacer campos.

\item {} 
Eliminar vértices duplicados.

\item {} 
Insertar registros a la capa.

\sphinxincludegraphics{{gc_terreno}.png}

\end{enumerate}


\subsection{GC\_Comisiones\_Terreno}
\label{\detokenize{ETL_SNC/Model:gc-comisiones-terreno}}\begin{enumerate}
\sphinxsetlistlabels{\arabic}{enumi}{enumii}{}{.}%
\item {} 
Unir capas vectoriales.

\item {} 
Corregir geometrías.

\item {} 
Unir atributos por valor de campo.

\item {} 
Rehacer campos.

\item {} 
Eliminar vértices duplicados.

\item {} 
Insertar registros a la capa.

\sphinxincludegraphics{{gc_comisiones_terreno}.png}

\end{enumerate}


\subsection{GC\_Direccion}
\label{\detokenize{ETL_SNC/Model:gc-direccion}}\begin{enumerate}
\sphinxsetlistlabels{\arabic}{enumi}{enumii}{}{.}%
\item {} 
Unir atributos por valor de campo.

\item {} 
Unir atributos por valor de campo.

\item {} 
Unir capas vectoriales.

\textendash{}Ruta 1

\item {} 
Unir atributos por valor de campo.

\item {} 
Unir atributos por valor de campo.

\item {} 
Rehacer campos.

\textendash{}Ruta 2

\item {} 
Extraer por atributo.

\item {} 
Rehacer campos.

\item {} 
Insertar registros a la capa.

\item {} 
Insertar registros a la capa.

\sphinxincludegraphics{{gc_direccion}.png}

\end{enumerate}


\subsection{GC\_Construccion}
\label{\detokenize{ETL_SNC/Model:gc-construccion}}\begin{enumerate}
\sphinxsetlistlabels{\arabic}{enumi}{enumii}{}{.}%
\item {} 
Unir capas vectoriales.

\item {} 
Corregir geometrías.

\item {} 
Unir atributos por valor de campo.

\item {} 
Rehacer campos.

\item {} 
Eliminar vértices duplicados.

\item {} 
Insertar registros a la capa.

\sphinxincludegraphics{{gc_construccion}.png}

\end{enumerate}


\subsection{GC\_Comisiones\_Construccion}
\label{\detokenize{ETL_SNC/Model:gc-comisiones-construccion}}\begin{enumerate}
\sphinxsetlistlabels{\arabic}{enumi}{enumii}{}{.}%
\item {} 
Unir capas vectoriales.

\item {} 
Corregir geometrías.

\item {} 
Unir atributos por valor de campo.

\item {} 
Rehacer campos.

\item {} 
Eliminar vértices duplicados.

\item {} 
Insertar registros a la capa.

\sphinxincludegraphics{{gc_comisiones_construccion}.png}

\end{enumerate}


\subsection{GC\_Unidad\_Construccion}
\label{\detokenize{ETL_SNC/Model:gc-unidad-construccion}}
Nota: la parte superior hace referencia a la parte geográfica y la inferior a la alfanumérica.

\begin{sphinxVerbatim}[commandchars=\\\{\}]
\PYG{n}{Parte} \PYG{n}{Superior}  
\end{sphinxVerbatim}
\begin{enumerate}
\sphinxsetlistlabels{\arabic}{enumi}{enumii}{}{.}%
\item {} 
Unir capas vectoriales.

\item {} 
Corregir geometrías.

\item {} 
Unir atributos por valor de campo.

\item {} 
Rehacer campos.

\end{enumerate}

Parte Inferior
\begin{enumerate}
\sphinxsetlistlabels{\arabic}{enumi}{enumii}{}{.}%
\item {} 
Calculadora de campos.

\item {} 
Calculadora de campos.

\item {} 
Extraer por atributo x 3 (tres).

\item {} 
Unir atributos por valor de campo x 3 (tres).

\item {} 
Extraer por atributo x 3 (tres).

\item {} 
Rehacer campos x 3 (tres).

\item {} 
Unir capas vectoriales.

\item {} 
Eliminar vértices duplicados.

\item {} 
Insertar registros a la capa.

\sphinxincludegraphics{{gc_unidad_construccion}.png}

\end{enumerate}


\subsection{GC\_Comisiones\_Unidad\_Construccion}
\label{\detokenize{ETL_SNC/Model:gc-comisiones-unidad-construccion}}\begin{enumerate}
\sphinxsetlistlabels{\arabic}{enumi}{enumii}{}{.}%
\item {} 
Unir capas vectoriales.

\item {} 
Corregir geometrías.

\item {} 
Unir atributos por valor de campo.

\item {} 
Rehacer campos.

\item {} 
Eliminar vértices duplicados.

\item {} 
Insertar registros a la capa.

\sphinxincludegraphics{{gc_comisiones_unidad_construccion}.png}

\end{enumerate}


\subsection{GC\_Propietario}
\label{\detokenize{ETL_SNC/Model:gc-propietario}}\begin{enumerate}
\sphinxsetlistlabels{\arabic}{enumi}{enumii}{}{.}%
\item {} 
Corregir geometrías.

\item {} 
Rehacer campos.

\item {} 
Eliminar vértices duplicados.

\item {} 
Insertar registros a la capa.

\sphinxincludegraphics{{gc_propietario}.png}

\end{enumerate}


\subsection{GC\_Perimetro}
\label{\detokenize{ETL_SNC/Model:gc-perimetro}}\begin{enumerate}
\sphinxsetlistlabels{\arabic}{enumi}{enumii}{}{.}%
\item {} 
Corregir geometrías.

\item {} 
Rehacer campos.

\item {} 
Eliminar vértices duplicados.

\item {} 
Insertar registros a la capa.

\sphinxincludegraphics{{gc_perimetro}.png}

\end{enumerate}


\subsection{GC\_Vereda}
\label{\detokenize{ETL_SNC/Model:gc-vereda}}\begin{enumerate}
\sphinxsetlistlabels{\arabic}{enumi}{enumii}{}{.}%
\item {} 
Corregir geometrías.

\item {} 
Rehacer campos.

\item {} 
Eliminar vértices duplicados.

\item {} 
Insertar registros a la capa.

\sphinxincludegraphics{{gc_vereda}.png}

\end{enumerate}


\subsection{GC\_Manzana}
\label{\detokenize{ETL_SNC/Model:gc-manzana}}\begin{enumerate}
\sphinxsetlistlabels{\arabic}{enumi}{enumii}{}{.}%
\item {} 
Corregir geometrías.

\item {} 
Rehacer campos.

\item {} 
Eliminar vértices duplicados.

\item {} 
Insertar registros a la capa.

\sphinxincludegraphics{{gc_manzana}.png}

\end{enumerate}


\subsection{GC\_Barrio}
\label{\detokenize{ETL_SNC/Model:gc-barrio}}\begin{enumerate}
\sphinxsetlistlabels{\arabic}{enumi}{enumii}{}{.}%
\item {} 
Corregir geometrías.

\item {} 
Rehacer campos.

\item {} 
Eliminar vértices duplicados.

\item {} 
Insertar registros a la capa.

\sphinxincludegraphics{{gc_barrio}.png}

\end{enumerate}


\subsection{GC\_Sector\_Urbano}
\label{\detokenize{ETL_SNC/Model:gc-sector-urbano}}\begin{enumerate}
\sphinxsetlistlabels{\arabic}{enumi}{enumii}{}{.}%
\item {} 
Corregir geometrías.

\item {} 
Rehacer campos.

\item {} 
Eliminar vértices duplicados.

\item {} 
Insertar registros a la capa.

\sphinxincludegraphics{{gc_sector_urbano}.png}

\end{enumerate}


\subsection{GC\_Sector\_Rural}
\label{\detokenize{ETL_SNC/Model:gc-sector-rural}}\begin{enumerate}
\sphinxsetlistlabels{\arabic}{enumi}{enumii}{}{.}%
\item {} 
Corregir geometrías.

\item {} 
Rehacer campos.

\item {} 
Eliminar vértices duplicados.

\item {} 
Insertar registros a la capa.

\sphinxincludegraphics{{gc_sector_rural}.png}

\end{enumerate}


\section{Estructura del ETL SNC en repositorio}
\label{\detokenize{ETL_SNC/Repository:estructura-del-etl-snc-en-repositorio}}\label{\detokenize{ETL_SNC/Repository::doc}}
\sphinxhref{https://github.com/AgenciaImplementacion/Asistente-LADM-COL}{Repositorio}

\sphinxhref{https://github.com/AgenciaImplementacion/Asistente-LADM-COL/tree/master/asistente\_ladm\_col/gui/supplies}{Lógica de la interfaz}
\begin{itemize}
\item {} 
snc\_data\_sources\_widget.py

\item {} 
wiz\_supplies\_etl.py

\end{itemize}

\sphinxhref{https://github.com/AgenciaImplementacion/Asistente-LADM-COL/tree/master/asistente\_ladm\_col/resources/etl}{Recursos}
\begin{itemize}
\item {} 
ficha\_matriz.csv

\item {} 
ficha\_matriz\_predio.csv

\item {} 
ficha\_matriz\_torre.csv

\item {} 
predio\_sancion.csv

\item {} 
unidad\_construccion.csvt

\end{itemize}

\sphinxhref{https://github.com/AgenciaImplementacion/Asistente-LADM-COL/tree/master/asistente\_ladm\_col/ui/supplies}{Interfaces}
\begin{itemize}
\item {} 
snc\_data\_source\_widget.ui

\item {} 
wiz\_supplies\_etl.ui

\end{itemize}



\renewcommand{\indexname}{Índice}
\printindex
\end{document}